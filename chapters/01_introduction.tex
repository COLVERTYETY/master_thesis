
\section{Introduction}

\subsection{Background}

In 2007, the launch of the iPhone marked a pivotal moment not just in mobile technology, but in human-computer interaction (HCI).
One of its most innovative features was the virtual keyboard. At a time when physical keyboards dominated the mobile market, the iPhone’s touch screen keyboard presented a radical shift.
Crucially, this keyboard utilized predictive text algorithms, an early application of artificial intelligence (AI) in consumer products, to improve user experience on a smaller screen.
This AI-driven approach did not just correct mistakes but anticipated user intentions, setting a new standard for HCI.

Since the inception of the iPhone's keyboard, the landscape of AI has undergone profound transformations, particularly with the advent of deep learning technologies.
These advancements have not only enhanced existing interfaces but have also paved the way for new interaction paradigms.
Today, AI is not merely a tool for enhancing user interfaces but is central to the design and implementation of innovative HCI solutions.
It is capable of understanding and predicting user behavior, facilitating more natural and intuitive interactions between humans and machines.

\subsection{Research Domain}

This thesis aims to explore the new paradigm introduced by AI in HCI and how it enables low-signaling, high-possibility affordances for both user interfaces (UI) and user experiences (UX).
The concept of affordances in HCI refers to the potential actions that an interface offers to users, influenced by the cues and signals embedded in the design.
AI, with its ability to infer and predict user needs, introduces a shift where affordances can transcend explicit cues, creating opportunities for seamless and dynamic interactions.

\subsection{Problem Statement}

The integration of AI into HCI has transformed traditional affordances, enabling interactions that are more dynamic, predictive, and adaptive.
However, this shift raises questions about how these AI-driven affordances differ from traditional ones and how they impact user interaction.
There is a need to understand both the opportunities and challenges that arise from designing interfaces that rely on low-signaling, high-possibility affordances.

\subsection{Objectives}

The objectives of this research are:
\begin{itemize}
    \item To analyze how AI can foster more intuitive and expressive interactions within HCI.
    \item To investigate the challenges that arise when designing for low-signaling, high-possibility affordances.
    \item To explore how AI-driven affordances can enhance or hinder user interaction.
\end{itemize}

\subsection{Research Questions}

This thesis seeks to answer the following key research questions:

\begin{itemize}
    \item How do AI-driven affordances differ from traditional affordances in HCI?
    \item In what ways can AI-driven affordances enhance user interaction?
    \item What challenges are associated with designing for low-signaling, high-possibility affordances?
\end{itemize}

\subsection{Significance}

The significance of this research lies in its potential to influence how future interfaces are designed.
By examining the evolving role of AI in HCI, this thesis seeks to contribute to the understanding of how AI can enhance user autonomy, creativity, and engagement.
The findings could have implications for designers and engineers striving to create more responsive and adaptive systems that leverage AI to enrich the human experience.

\subsection{Projects Overview and Contributions}

\subsubsection{Totem Project}

The Totem project explores the integration of AI and augmented reality (AR) in the realm of digital art.
Users interact with the system using hand movements in front of a screen to generate and manipulate digital art in real-time.
Powered by SDXL Turbo, the project uses deep learning to generate portraits, while the First Order Motion model adds dynamic motion, enabling a fluid creative process where users engage with AI as a collaborator.
This interaction is facilitated through MediaPipe, which allows real-time hand tracking, transforming gestures into creative input.

\subsubsection{LLM White Board}

The LLM white board is designed to explore user collaboration with a large language model (LLM) in a live programming environment.
This project allows users to explore physical computing concepts in AR, with spatial paradigms enabled by AI.
Cameras capture the physical environment, while models interpret the data, and the LLM generates code based on user commands.
The white board environment, which can be a blank canvas or overlaid on a camera feed, uses MediaPipe for object and user tracking, providing an adaptive and interactive interface that blurs the boundary between the digital and physical worlds.