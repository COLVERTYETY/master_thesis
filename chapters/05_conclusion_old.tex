\section{Conclusion}

\subsection{Revisiting the Research Questions}

This thesis addressed core questions on integrating AI and AR in HCI, focusing on low-signaling, high-possibility affordances and AI-driven collaboration.

\subsubsection{Challenges in Low-Signaling Affordances}
Designing low-signaling interfaces requires balancing subtle guidance with user autonomy.
The Totem platform exemplified this by enabling users to explore interactive digital art with minimal gestures, supported by responsive, context-aware AI cues.
The LLM Whiteboard further demonstrated adaptive, spatially contextual affordances, guiding users intuitively through high-possibility interactions with reduced visual clutter.

\subsubsection{AI and User Collaboration}
Both Totem and LLM Whiteboard showcased AI’s role as a collaborative partner.
The Totem empowered users to create intricate digital art without technical expertise, while the LLM Whiteboard supported real-time coding, interpreting commands and autonomously refining outputs.
These projects highlight AI's ability to enhance creative tasks, assisting users without overshadowing their intent.

\subsubsection{Differences from Traditional Affordances}
AI-driven affordances shift from static signals (e.g., icons) to dynamic, context-sensitive responses.
The Totem’s real-time, gesture-based interactions, powered by MediaPipe, exemplify this responsive, adaptable approach.
Similarly, the LLM Whiteboard used high-level commands and natural language to create fluid, interpretive interactions, moving beyond traditional cues and embracing co-creative engagement with AI.

\subsection{Contributions to the Field}

This thesis advances AI-driven AR and HCI by demonstrating how AI enhances creativity and redefines interaction paradigms.

\subsubsection{AI-Driven Creativity}
The Totem platform democratizes digital art creation, allowing users to generate and animate high-quality content with simple sketches and gestures.
Using SDXL Turbo for image generation and First Order Motion for animation, the Totem enables an intuitive, interactive creative process.
By lowering skill barriers, the Totem illustrates how AI can become a collaborative partner in creativity, seamlessly integrating digital art creation into user interactions.

\subsubsection{New HCI Paradigms}
The LLM Whiteboard introduces a novel HCI model, using high-level commands and spatial affordances to facilitate AI-driven, low-signaling interactions.
Its dual modes—open canvas coding and AR for spatial engagement—allow users to generate complex outputs with minimal input.
By shifting away from explicit cues, the LLM Whiteboard exemplifies dynamic, adaptive interactions, setting a foundation for intuitive, AI-augmented environments that bridge digital and physical spaces.

\subsection{Future Implications}

This research outlines pathways for AI-enabled AR, with implications for collaborative creativity, educational accessibility, and ethical considerations in user autonomy.

\subsubsection{Collaborative Creativity Across Domains}
The Totem and LLM Whiteboard platforms reveal new possibilities for enhancing creative workflows in fields like digital art, coding, and design.
Future AI-augmented systems could expand upon these models, enabling dynamic, responsive collaborations where AI tailors assistance to user preferences.
By fostering AI as an adaptive co-creator, future platforms may redefine digital collaboration, making complex creative outputs achievable with minimal input.

\subsubsection{Advances in Educational Accessibility}
This thesis highlights the potential of low-barrier, intuitive interfaces to democratize access to digital tools, making creative technologies approachable for diverse users.
AR-driven educational applications could introduce students to complex tasks like digital art and programming through interactive, hands-on experiences.
Such systems align with trends in HCI that emphasize accessibility, supporting inclusive learning across varied environments and empowering a broader audience to create, learn, and innovate.

\subsubsection{Balancing Autonomy with AI-Driven Guidance}
As AI integrates into creative and educational tools, balancing user control with AI guidance is crucial.
Totem and LLM Whiteboard exemplify AI’s supportive role, enhancing user experience while preserving autonomy.
Future research should explore this balance, focusing on transparent AI interactions that empower users without overshadowing their contributions.
Thoughtful design and clear communication of AI’s role can maintain user trust and prevent over-reliance on automation, ensuring AI remains a facilitator of, not a replacement for, human creativity.

\subsection{Limitations and Future Work}

This thesis advances AI-driven AR in HCI but also reveals limitations that prompt further investigation to improve technical performance, user experience, and ethical standards.

\subsubsection{Technical Constraints}
A primary limitation is the processing capacity of edge devices, which impacts real-time interactivity for platforms like Totem and LLM Whiteboard.
Despite optimizations with frameworks like ONNX, edge devices still struggle with high-resolution graphics and complex computations, limiting scalability on consumer hardware.

Future work could explore cloud computing and distributed processing to offload intensive tasks, creating more responsive experiences without taxing user devices.
Hybrid models balancing local and cloud processing, along with hardware-optimized AI models, may enable scalable, accessible real-time AR.

\subsubsection{User Interaction Challenges}
Low-signaling, high-possibility affordances present a usability challenge, as limited cues may hinder navigation, especially for users unfamiliar with minimalistic designs.
This could restrict adoption across broader audiences.

Future research might address this with adaptive signaling that responds to user behavior or context, offering guidance without cluttering the interface.
Such adaptivity could bridge the gap between minimalist design and ease of use, enhancing inclusivity in low-signaling interfaces.

\subsubsection{Ethics and User Autonomy}
Balancing AI collaboration with user autonomy is essential, especially as AI takes on a significant role in creative outputs.
Totem and LLM Whiteboard highlight the importance of preserving authorship and agency, but blurred boundaries between user and AI contributions present ethical challenges.

Future research could focus on transparent interface design that clarifies AI’s role in creation, ensuring users understand where AI input starts and ends.
Establishing ethical guidelines for AI-driven creativity will be vital to promoting AI as an empowering tool rather than a replacement, supporting creative autonomy.

\subsection{Final Thoughts}

This research marks a step forward in realizing a creative future powered by AI-driven augmented reality.
Through the development of the Totem and LLM Whiteboard platforms, this work illustrates the potential for AI to enhance not only creative expression but also the user experience in digital environments.
These platforms demonstrate how AI can simplify complex tasks, provide users with intuitive, low-barrier interfaces, and invite collaborative creativity that bridges the digital and physical worlds.

At the core of this vision is the transformative convergence of AI and HCI, which promises to reshape personal and professional landscapes alike.
By merging advanced machine learning models with responsive, real-time AR environments, this thesis highlights how AI can go beyond functioning as a mere tool to become an active partner in creation, education, and interaction.
The Totem and LLM Whiteboard showcase new paradigms where users are empowered to explore, collaborate, and create in ways that feel as intuitive as they are powerful, setting a new standard for future developments in HCI.

Looking ahead, the implications of this research extend far beyond the applications explored here.
As AR and AI technologies continue to advance, the concepts explored in this work—intuitive affordances, adaptive interfaces, and collaborative AI—hold the promise of transforming our interactions with technology, making it a supportive extension of our own creativity and cognition.
With continued innovation, the future of AI-driven HCI is not only exciting but poised to make digital creativity more responsive to the needs of diverse users.

In conclusion, this thesis lays a foundation for an AI-augmented future where technology is an enabler of human potential, enriching our experiences and expanding the horizons of what we can create.
As AI and AR continue to evolve, the possibilities are limitless, driven by the shared goal of making our interactions with technology as empowering and expressive as possible.
The journey toward this future is just beginning, and the work presented here offers a glimpse of what lies ahead—a world where AI and AR transform our digital experiences, making creativity accessible to all and limited only by imagination.