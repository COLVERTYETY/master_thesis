\section{Conclusion}
The exploration of affordances in human-computer interaction (HCI) through the lens of artificial intelligence (AI) has revealed a paradigm shift in how interfaces are designed and how users interact with them.
Traditionally, affordances were driven by explicit visual or tactile cues that guided users to perform specific actions.
However, with the integration of AI, a new interaction model emerges—one that prioritizes low-signaling, high-possibility affordances, allowing for a more dynamic, adaptive, and user-centric experience.

Through this thesis, we have investigated the evolving role of AI in HCI, particularly how AI-driven affordances transcend explicit cues, enabling seamless and intuitive interactions.
The Totem Project and LLM Whiteboard exemplify the core ideas of this research, demonstrating how AI can be employed to reduce the complexity of user input while enhancing the possibilities for interaction.
Both projects showcase how AI enables interfaces to predict, infer, and respond to user behavior in ways that are context-aware and responsive to creative expression.

\subsection{Revisiting the Research Questions}
The central research question of how AI-driven affordances differ from traditional affordances has been answered through the study of AI’s ability to infer intent with minimal explicit guidance.
Unlike traditional affordances, which rely on clear, fixed signals to indicate possible actions, AI-driven affordances adapt based on real-time user interactions, creating a more flexible and intuitive user experience.
For example, in the Totem Project, AI interprets user gestures and transforms basic sketches into interactive, dynamic art with minimal input.
This demonstrates that AI-driven affordances expand the creative possibilities within an interface, fostering more fluid and expressive user interactions.

The research has also shown how AI-driven affordances can enhance user interaction by offering a greater range of possibilities with less cognitive effort from the user.
The LLM Whiteboard project highlights how large language models (LLMs) act as semantic operators, allowing users to give high-level commands and generate complex outputs with ease.
This reflects how AI-driven systems enable low-signaling, high-possibility interactions, where users are empowered to engage with interfaces on an abstract level without needing to manage every detail.

However, challenges associated with designing for low-signaling, high-possibility affordances were also identified.
One of the primary concerns is that users may struggle to understand the full scope of available actions if the signaling is too minimal.
This challenge was explored in both projects, where balancing user guidance and freedom became crucial.
For instance, in augmented reality (AR) environments like AI Totem, the risk of overwhelming users with possibilities without clear affordances was addressed by embedding adaptive AI cues that only surface when needed.

\subsection{Contributions to the Field}
This thesis contributes to the growing body of knowledge on how AI can enrich HCI by fostering more intuitive, autonomous, and creative interactions.
It builds on the idea that AI is not just a tool for enhancing interfaces but a central component of how interfaces are designed, moving beyond traditional affordance models toward more adaptive, predictive, and dynamic systems.

The Totem Project showcases how AI and AR can work in synergy to transform creative processes, turning user sketches into real-time, animated art.
This project serves as a model for how creative expression in digital environments can be enhanced through AI-assisted affordances that minimize technical barriers while maximizing artistic potential.

Similarly, the LLM Whiteboard provides a glimpse into the future of collaborative AI in HCI, where users and AI systems can work together to create complex outputs through high-level commands.
The project illustrates the potential of LLMs to act as intelligent collaborators, further extending the possibilities of low-signaling, high-possibility interfaces in both creative and technical domains.

\subsection{Future Implications}
The findings of this thesis have significant implications for the design of future HCI systems.
As AI continues to evolve, we can expect more interfaces to adopt the low-signaling, high-possibility paradigm, offering users a more seamless and engaging experience.
This shift toward AI-driven interaction models opens new possibilities for personalized, adaptive systems that cater to individual user needs while supporting a wide range of applications—from digital art and design to real-time collaboration in technical environments.

Ultimately, the research presented in this thesis advocates for a rethinking of affordance design in light of AI's capabilities.
By leveraging AI to balance signaling and possibility, designers and engineers can create more natural, fluid, and interactive experiences, aligning with the growing expectations of intuitive, AI-augmented systems in the modern world.

\subsection{Final Thoughts}
In conclusion, AI has emerged as a transformative force in HCI, enabling new forms of interaction that were previously unimaginable.
Through the lens of affordances, we have seen how AI can bridge the gap between user intention and system response, making interfaces more adaptive, dynamic, and creatively empowering. As this field continues to evolve, the principles explored in this thesis will serve as foundational concepts for the next generation of AI-enhanced user experiences.